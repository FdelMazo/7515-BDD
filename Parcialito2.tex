\documentclass[10pt]{article}

\usepackage[a4paper, left=1in, bottom=1in]{geometry}
\thispagestyle{empty}
\usepackage{pdfoverlay}
\usepackage[normalem]{ulem}
\pdfoverlaySetPDF{Parcialito2-Enunciado.pdf}

\begin{document}

\begin{picture}(0,0)
	\put(410,12){100029}
	\put(405,-12){del Mazo}
	\put(406,-35){Federico}
\end{picture}

\vspace{35em}
Entidades pasadas a relaciones:

\begin{itemize}
	\item A(\underline{id\_A}, a1)
	\item B(\underline{id\_B}, b1)
	\item C(\underline{id\_C}, c1)
	\item D(\underline{id\_D}, d1)
	\item E(\underline{\dashuline{id\_C}}, e1, e2)
	\item F(\underline{\dashuline{id\_C}}, f1, f2)
\end{itemize}

Interrelaciones pasadas a relaciones:

\begin{itemize}
	\item R1(\underline{\dashuline{id\_C, id\_C}})
	\item R2(\underline{\dashuline{id\_D, id\_C}}, r1)
	\item R3(\dashuline{id\_A, id\_B, id\_D})
	\begin{itemize}
		\item En esta relación, los 3 pares son una clave candidata: \\ \{ \{ id\_A, id\_B\}, \{ id\_B, id\_D\}, \{ id\_A, id\_D\} \}
		\item Hay que tomar uno de estos 3 pares como clave primaria, es indistinto cual tomemos (al menos es indistinto en este modelo abstracto, tal vez en un dominio especializado encontremos una lógica detrás de la elección)
	\end{itemize}
\end{itemize}

Claves primarias \underline{subrayadas}, claves foráneas \dashuline{subrayadas a rayas}, claves candidatas (adicionales a la clave primaria) especificadas

\end{document}
